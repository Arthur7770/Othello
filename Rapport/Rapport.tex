\documentclass[12pt]{article}

\usepackage[utf8]{inputenc}
\usepackage[T1]{fontenc}
\usepackage[french]{babel}
\usepackage{hyperref}
\usepackage{graphicx}
\title{Aide à la Décision - Othello}
\author{Antonin Boyon \and Quentin Legot \and Arthur Page}
\date{\today}

\begin{document}

\maketitle
\thispagestyle{empty}
\setcounter{page}{0}
\newpage

\tableofcontents
\newpage

\section{Introduction}
Le but de notre projet était de concevoir un algorithme de recherche performant sur un jeu d' \textit{Othello}. Le jeu est le plus abstrait possible, la partie nous intéressant étant la réalisation d'un algorithme de recherche efficace. Il est ainsi impossible de jouer au jeu, on ne peut que regarder le résultat d'une partie entre deux joueurs artificiels.
Une fois le jeu et l'algorithme de recherche implémentés, nous serons en mesure d'analyser ce dernier pour définir ses paramètres de fonctionnement optimaux. Nous aborderons dans un premier temps l'implémentation du jeu, puis celle de l'algorithme et enfin la présentation et l'analyse des mesures observées.

\section{Le jeu}


\section{L'algorithme de recherche}
\subsection{Algorithme de base}
\subsection{Algorithme d'élagage}

\section{Mesures}
\subsection{Présentation des mesures}
\subsection{Analyse des mesures}

\section{Conclusion}
\end{document}